\documentclass{ltxdockit}[2011/03/25]
\usepackage{btxdockit}
\usepackage{fontspec}
\usepackage[mono=false]{libertine}
\usepackage{microtype}
\usepackage[american]{babel}
\usepackage[strict]{csquotes}
\setmonofont[Scale=MatchLowercase]{DejaVu Sans Mono}
\usepackage{shortvrb}
\usepackage{pifont}
\usepackage{minted}
\usepackage{bidi}
% Meta-datas
\titlepage{%
	title={Handout fabrication},
	subtitle={New data types},
	email={maieul <at> maieul <dot> net},
	author={Maïeul Rouquette},
	revision={1.0.0},
	date={20/03/2014},
	url={https://github.com/maieul/handout}}


\begin{document}

\printtitlepage
\tableofcontents

\section{Introduction}
\subsection{Goals}

In some scholar fields, a beamer is not a good support when speeching in a proceeding. For example in classical philology, as the main sources are text, it will be better to distribute to the auditors an handout with extracts of the texts about which we will speech.

The goal of this package is to help to prepare such handout when writting the talk.

\subsection{Credits}

This package was created for Maïeul Rouquette's PHD\footnote{\url{http://apocryphes.hypothese.org}.} in 2014. It is licenced on the \emph{\LaTeX\ Project Public Licence}\footnote{\url{http://latex-project.org/lppl/lppl-1-3c.html}.}.

All issues can be submitted, in French or English, in the GitHub issues page\footnote{\url{https://github.com/maieul/handout/issues}.}.

\subsection{French tutorial}

As the idea of the package behavior was send by French \LaTeX\ users\footnote{\url{http://fr.comp.text.tex.narkive.com/pXMop2kE/fabrication-d-un-exemplier}.}, and as the package's author has French native language, a French tutorial is available in \url{http://geekographie.maieul.net/136}. 

All files in the examples' folder are in French, but that should not have consequence for the meaning of their behaviors. We  have to be run with \XeLaTeX.


\section{Basis}

The package can be loaded very quickly with the standard command \cs{usepackage}

\begin{minted}{tex}
\usepackage{handout}
\end{minted}
The idea of the this package is to prepare handout during the writing of the paper. When you want to add something in your handout, just write it on an external file, and call this external file with the command \cs{handout} :

\begin{minted}{tex}
Your text
\handout{folder/example}
Your text
\end{minted}

The PDF output will contain two parts:
\begin{enumerate}
	\item Your paper.
	\item The handout.
\end{enumerate}

You have just to split your pdf in two parts to obtain your handout for auditors.

See example~1.


\section{Putting all the examples' file in a same folder}

In most cases, your file will be put in the same folder. So you can fix this folder with the package's option \opt{dir}:
\begin{minted}{tex}
\usepackage[dir=folder]{handout}
...
Your text
\handout{example}
Your text
\end{minted}{tex}

See example~3.

\section{Change history}

\begin{changelog}

\begin{release}{1.0.0}{2014-03-20}
\item First public release.
\end{release}
\end{changelog}
\end{document}
